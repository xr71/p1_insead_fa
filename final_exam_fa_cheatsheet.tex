\documentclass[8pt]{report}
\usepackage{multicol}
\usepackage[a4paper, portrait, margin=0.5in]{geometry}
\setlength\parindent{0pt}
\setlength{\abovedisplayshortskip}{-1pt}
\setlength{\belowdisplayshortskip}{0pt} 
\usepackage{graphicx}
\usepackage{mathtools}
\pagenumbering{gobble}

\begin{document}

Company - Investors and Customers: Investors are debtholders plus equityholders. Accounting is for the communication of information to investors - information about the underlying economics 

\begin{multicols}{3}
[
\section{Financial Reporting Process}
Balance Sheet, Income Statement, and Financial Ratios
]

Balance Sheet = Stock "snapshot"
\begin{itemize}
	\item Balance Sheet describes Financial Position
	\item $A = L+E$
\end{itemize}
Income Statement = $\Delta$stock or the Flow
\begin{itemize}
	\item Financial performance over a period of time
	\item Operating Income = benefits \& sacrafices due to operating
	\item Pre-Tax Income = benefits \& sacrafices due to financing \& investing
	\item NET INCOME = bottom line
\end{itemize}

Analytical Tools:
\begin{enumerate}
\item Journal Entries
\item T-Accounts
\item BSE Template
\end{enumerate}

Left and Right:
\begin{itemize}
	\item Lea - LEFT
	\begin{itemize}
		\item Expenses
		\item Assets
	\end{itemize}
	\item lieR
	\begin{itemize}
		\item liability
		\item equity
		\item income
	\end{itemize}
\end{itemize}
\end{multicols}

\hrule

\begin{multicols}{3}
Accounting Information = Underlying Economics + Accounting Rules \& Principles + Manager Opportunism

$$
	Usefulness = Relevance \cdot Reliability
$$

\textbf{Assets}
\begin{itemize}
	\item economic resources
	\item owned or controleld by company
	\item probable future economic benefits
	\item result of past transactions or events
\end{itemize}

\textbf{Liabilities}
\begin{itemize}
	\item claims to economic resources by creditors
	\item economic obligations paybable to outsiders
	\item probable future economic sacrafices of resources
	\item results of past transactions or events
\end{itemize}


\textbf{Equity}
\begin{itemize}
	\item residual claim to economic resources
	\item paid what remains, so riskier than debt, but commands higher returns
\end{itemize}
\end{multicols}

\hrule


\begin{multicols}{2}
Ratios

$$
	Gross Profit Margin = \frac{Gross Profit}{Revenues}
$$
Where gross profit = $Revenues - COGS$
$$
	Operating Profit Margin = \frac{Operating Profit}{Revenues}
$$
Where operating profit = $Revenues - COGS - Operating Expenses$
$$
	Profit Margin = \frac{Net Income}{Revenues}
$$
Where net income = $Revenues - COGS - Operating Expenses \pm Other Gains/Losses$ \\


\textbf{Efficiency Ratios}
\begin{align*}
&	Asset \; Turnover = \frac{Revenues}{Avg Total Assets} \\
&	AR \; Turnover = \frac{Revenues}{Avg AR} \\ 
&	AP \; Turnover = \frac{COGS}{Revenues} \\
& 	Inv \; Turnover = \frac{COGS}{Avg Inventory} \\
\end{align*}

AND 
\begin{align*}
&	Days in AR = \frac{365}{AR Turnover} \\
& 	Days in AP = \frac{365}{AP Turnover} \\
&	Inv Holding Period = \frac{365}{Inv Turnover}
\end{align*}

\textbf{Liquidity Ratios}
\begin{align*}
&	Current Ratio = \frac{Current Assets}{Current Liabilities} \\
& 	Quick Ratio = \frac{Current Assets - Inventory}{Current Liabilities}
\end{align*}

\end{multicols}

\begin{multicols}{2}
[
\section{Earnings vs Cash Flow}
Investors invest money in companies because they want a return on their invested capital.
]
Retained Earnings
$$
	Ending RE = Beginning RE + Net Income - Dividends
$$

ROIC:
$$
	ROIC = \frac{Return}{InvestedCapital}
$$
Where IC = $OperatingAssets - OperatingLiabilities$ \\
and IC = $\frac{beginning NOA + ending NOA}{2}$ \\
NOPAT: 
$$
	NOPAT = Operating_{rev} - NonTax_{OperatingExpenses} - Operating_{taxes}
$$

$$
	Profit Margin = \frac{NOPAT}{REVENUE}
$$

$$
	ICTurnover = \frac{Revenue}{AvgInvestedCapital}
$$

Hence $$
	ROIC = ProfitMarge \cdot ICTurnover
$$
\end{multicols}

\hrule 


\begin{multicols}{2}
FixedAssetTurnover
$$
	\frac{Revenue}{Avg Tanginle Fixed Assets}
$$


\end{multicols}


\begin{multicols}{2}
[
\section{Operating Activities}
Balance Sheet, Income Statement, and Financial Ratios
]


\end{multicols}


\begin{multicols}{2}
[
\section{Investing \& Financing Activities}
Balance Sheet, Income Statement, and Financial Ratios
]


\end{multicols}

\end{document}